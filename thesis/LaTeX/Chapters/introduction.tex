% Background & Motivation
% Problem Statement
% Objective
% Scope & Limitation
% Overview of structure

\chapter{Introduction}\label{ch:introduction}
In today's rapidly evolving world, the field of aerospace engineering plays a crucial role in the development of
cutting-edge technologies that facilitate travel, communication, and exploration.
A key aspect of aircraft system engineering is the emphasis on safety and redundancy concepts,
which ensure the integrity and reliability of critical components, and by extension, the safety of the passengers and crew.
The effective dissemination of these principles to the younger generation is essential for inspiring and cultivating future
talent in the aerospace industry.
This master's thesis seeks to address this need by designing and implementing an interactive,
engaging game specifically targeted at children, aimed at teaching and presenting them the basics of
aircraft system engineering with a focus on redundancy concepts and safety critical systems.
\\
\\
Aircraft system engineering encompasses a wide range of disciplines and concepts, many of which can be challenging for young learners to grasp.
By introducing these topics through the medium of a game,
this thesis aims to make the subject more accessible and enjoyable,
thereby fostering a greater understanding of the importance of safety and redundancy in aviation.
It is widely recognized that gamification can significantly enhance the learning experience,
and the adoption of this approach in this project is expected to be highly effective,
especially in engaging children in the topic and maintaining their interest throughout the learning process.
\\
\\
The foundation for this thesis is layed out by introducing essential concepts in system engineering such as redundancy and markov processes.
Furthermore, a brief overview about the game design process and design principles, game engines and games in education is given to
provide the necessary background knowledge for understanding and following the subsequent content and the overall context.
A conceptualization of a game engine and the game implemented by using the engine is conducted, which is outlined by the initial
design and planning stages of the game, including the identification of learning objectives, game mechanics and goals.
Strategies, architectures and approaches to the game and engine development are provided, highlighting the key considerations for both
technical and graphical design aspects.
The development and implementation of the game engine and game are described in detail, by delving into the programming and implementation of
different features available to both engine and game.
It includes a thorough analysis of the game's functionality and aesthetics, and provides insight into the game's potential impact on children's
engagement and understanding in the topic of aircraft system engineering with a focus on safety critical systems and redundancy concepts.
The accomplishments, limitations and challenges during the development will be provided alongside presenting prospects for future work, such
as refining the game based on potential user feedback, possible performance optimizations and the possibility of incorporating further requirements or
concepts to expand the game.
\\
\\
Ultimately, this project strives to achieve a delicate balance between entertainment and education,
aiming to create a game that is both engaging and informative for children.
By successfully implementing this gamified simulation of safety-relevant aircraft systems,
the goal is to spark curiosity and enthusiasm for the field of aerospace engineering,
laying the foundation for the next generation of innovative and safety-conscious professionals in the industry.