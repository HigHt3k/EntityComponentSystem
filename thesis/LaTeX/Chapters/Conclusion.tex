\chapter{Conclusion \& Prospects}\label{ch:conclusion-&-prospects}
In conclusion, this thesis has presented the development of a gamified simulation of safety relevant aircraft systems.
The motivation for this work was to provide a novel approach to gaining attention and interest for the topic of aircraft
system engineering, specifically at the University of Stuttgart, by combining an engaging and fun experience of gaming and the
informative content of basic system engineering concepts in aviation.
\\
The main contributions to this work are the conceptualization and development of a game engine that is used to
design and implement a game by using it.
Different scenarios are implemented to give insight into a range of different concepts, such as redundancy or different failure types in an
interactive environment for the target audience of younger children, teenagers or families to gain interest in a playful way.
The game itself uses a simplified simulation approach in the backend by using the Markov method to generate and simulate failure states
across the system in a way, that is also used in safety analysis of aircraft systems.
\\
The game engine follows a modular entity component system architecture, which enables flexible, maintainable and highly customizable
development of game entities and game behaviors.
This also allows for further development of the game and easy integration of new concepts, components, scenarios and logics.
Therefore, the game and its engine serve as a baseline to any further requirements that are made to further improve the experience of the game.
\\
As the first time this game is going to be presented to the public is the ``Tag der Wissenschaft'' at the University of Stuttgart, which is happening
on 13th of May 2023, there may be feedback which can be implemented in the game afterwards.
In case any misunderstandings happen during play-throughs of different levels or any bugs occur, that were not found during testing of
the application, a documentation of the former mentioned feedbacks should be helpful to give a guideline on any improvements that may be made.
\\
Additionally, it would be interesting to integrate a 2 player mode, which could either be cooperative in a way, that two players
try to build a system together, or competitive, where the screen can be split and a further factor for the score calculation may be added in the time difference
needed for each player to finish the game.
\\
As more people get involved in playing the game, this work serves as a basis for any further research, development and use-case integration in the area
of game-based learning and as a demonstrator game, which can contribute to making people interested in learning more about the topic and is possibly and effective
approach to gain future students and therefore future engineers in the field of aircraft system engineering.