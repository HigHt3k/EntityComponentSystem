\chapter{Conceptualization}\label{ch:design}
\section{Background}\label{sec:background}
\section{Genre}\label{sec:genre}
The \textit{Puzzle} genre is chosen for the graphical user interface and interaction, while the major calculations for the
games' content may be categorized as \textit{Simulation}.

\section{Intended Gameplay}\label{sec:intended-gameplay}
The gameplay is described for a singular level, however the basic concept is the same for each level.
A level usually describes an aircraft system, the goal of the game is to finish or repair a given system.
In order to do that, the user needs to choose from a variety of given components to build the system in a way, that the
required safety level will be reached.
This can be done in different ways, however the efficiency (i.e.\ how many components were used in the building process) determine
the rating upon passing a level.
Upon starting a level, the user will face a grid-based puzzle with some set components such as sensors, computers, actuators
or cables.
The user may add new components by choosing them from a limited stack of given components.
Components cannot interfere with other components, however components may also be removed or replaced.
As soon as the user presses a button to simulate the system, a simulation process in the backend is triggered to calculate
the safety rating of the system built.
Based on the safety rating calculated, an animation is triggered that shows the performance of the user visually by starting, cruising
and landing a plane.
If the safety rating is too low, the level is not passed and can be replayed or modified, if the rating is above the given
threshold, the level is passed and another level can be chosen.
A score is then calculated, depending on the efficiency of the system (i.e.\ how close the user got to the given safety requirement) and
how many of the available components were used.
The score can be saved to a scoreboard.
Upon passing a level, further levels are unlocked, which provides a certain amount of progression through the game.

\\

It is also possible to create new levels in a sandbox style environment, where the user can choose different pre-defined components and
safety requirements to create a custom level.

\section{Game Mechanics}\label{sec:game-mechanics}

\section{Graphic Design}\label{sec:graphic-design}
For graphics design, the graphic programs \textit{Krita}~\cite{foundation_2020} and \textit{Gimp}~\cite{gimp} are used.

\subsection{List of components}\label{subsec:list-of-components}
The following components need to be designed:
\begin{itemize}
    \item Screen backgrounds
    \item Background tiles
    \item Simulation components: actuator, sensor, computer, com-mon, cable, \ldots
    \item Build panel
    \item Requirement panel
\end{itemize}

\section{Sound Design}\label{sec:sound-design}
To give some atmosphere and create an 8 bit-styled experience, open source music will be used in the game.

\section{Language}\label{sec:language}
As the game is supposed to be used by a broad audience, especially in Germany and at the University Day of the University of Stuttgart, there should be an option
to change the language.
Since the audience is most likely to include young adults, teenagers and children, there are going to be multiple options even for a single language:
normal language and simplified language.
Furthermore, the game is designed to be understandable with a very low amount of reading text, however the text will always provide further information for interested people.
A language management system will be implemented which is based on using IDs (tags) instead of the actual text for any text string.
A processor is going to parse all available language files in XML format and a look-up table that maps each ID to the corresponding text string will be created.
Within the Rendering Engine, each text is going to be rendered in the language the game is set to.

\section{Level Design}\label{sec:level-design}
The level design is a main part of this work, as the progression through the game and the process of explaining different concepts
within aircraft system design is heavily influenced by this.
Therefore, levels were conceptualized before the implementation, to have a guideline for the actual implementation.
Levels are categorized in difficulties, reaching from tutorial to hard.
When passing a level, the user unlocks new content (e.g. levels) to progress through the game.
Tutorials should guide the user through the features of this game and provide some insight on the basics of system design, so
it can be used at later stages of the game.
A visual representation of this is needed to provide enough input for younger age-groups to play and understand the game.
However, for other users, some more in-depth knowledge is also provided in order to better understand calculations, simulations and
game logic in the background.
Tutorial levels explain the basic usage, such as cable and component (re-)placement and the different safety requirement
categories.
The categories are described by colors, from red (catastrophic) to blue (no safety effect).
Each component that can be built in the level also has this information to it, so the user can visually see a representation
of the criticality and the effect of the objects to the system and its requirements.
Each level goal is defined by a safety requirement and a minimum count of correctly operating actuators (or other components, such
as flight control computers, sensors, \ldots).
As the user progresses through the levels, there will be harder goals in safety requirements that need to be solved by using
redundancy concepts or mechanisms such as voting.
As this work aims to explain most of the basic concepts, some levels should also take into consideration common-mode and common-cause
failures.
Here, the main goal is to provide insight into the differentiation of these, which can be done by using sensors of different types.

\\ \\

The following levels have been designed for this game by default:
\begin{enumerate}
    \item Tutorial 01: Connecting cables to components
    \item Tutorial 02: Rotating cables
    \item Tutorial 03: Using multiple cables on a grid tile
    \item Tutorial 04: Placing components
    \item Tutorial 05: Replacing components on the grid
    \item Enhancing a Simplex System
    \item Cross-connecting of Duplex System
    \item Multiple Actuators
    \item Actuator with 2 required correct signals
    \item Duplex Sensor
\end{enumerate}
\\
\\
Additionally, through a build mode the user is able to set up new levels including the requirements / level goal via the built-in
frontend, which then saves the level to a new xml file.
