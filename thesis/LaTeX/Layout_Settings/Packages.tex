%% ============================================================
%%  ***  LaTeX packages
%% ============================================================

% Page layout
\usepackage{geometry}					% set page layout
\usepackage{setspace}					% set line spacing
\usepackage[automark]{scrlayer-scrpage}	% Koma header and footer package

% Language, coding and font
\usepackage{lmodern}					% Font - as requested by ILS
\usepackage[nenglish]{babel} 			% Language setting (last defined is standard)
\usepackage[T1]{fontenc}				% Hyphenation for words with Umlaute
\usepackage[utf8]{inputenc} 			% Coding for correct display of Umlaute
\usepackage[nenglish]{translator}		% Übersetzer
\usepackage{longtable}					% Tables with page break
\usepackage{tabu}
\usepackage{datetime}


% Graphics and colors
\usepackage[pdftex]{graphicx}					% Include graphics
\usepackage{epstopdf}							% Include eps graphics
\usepackage{color}								% Colors
\usepackage[svgnames,table,hyperref]{xcolor}	% Advanced colors

% Math
\usepackage{amsmath,amsthm}				% Math environment

% Floats
\usepackage[section]{placeins}			% Control float placement, \FloatBarrier command
% Other
\usepackage[hyphens]{url}				% URL
\usepackage{hyperref}					% Settings for PDF document
\usepackage{caption}					% for modification caption format
\usepackage{csquotes}					% Recommended
\usepackage{pdfpages}					% include PDF pages
\usepackage{lipsum}						% lorem ipsum blindtext
\usepackage{siunitx}					% si einheiten
\usepackage{microtype}					% underfull und overfull box problem minimierung
\usepackage{etoolbox}					% Appendix Buchstabenseitenzahl
\usepackage{float}						% In der Lage figures explixit an einer Stelle im Text zu fixieren
\usepackage[nottoc]{tocbibind}			% Abküzrungsverzeichnis, Tabellenverzeichnis
\usepackage{ulem}						% Underline text
\usepackage{paralist}					% Modifikation von Listen
\usepackage{titling}					% \theauthor macro

% Anpassbare Enumerates/Itemizes
\usepackage{enumitem}					% Bsp.: Option "style=nextline" für eine gleichmäßige Einrückung aller Zeilen


% Tabellen
\usepackage{lscape}				% mehrseitige Tabellen
\usepackage{booktabs}			% \toprule \midrule \bottomrule
\usepackage{colortbl}			% farbige Tabellen / Tabellen einfärben
\usepackage{multirow}			% mehrere Zeilen verbinden
\usepackage{array}				% Hilfsmittel zum Setzen von Tabellen und geordneten Texten im Mathematischem Modus

% Literatur
\usepackage
[backend=bibtex,						% Backends Bibtex
style=ieee,								% Bibliogragrafiestil IEEE
natbib=true]							% Kompatibilitätsmodul natbib
{biblatex} 								%
\addbibresource{Bibliography/BiB.bib}	% Dateipfad zur Bib Datei

% Glossaries
\usepackage[
xindy,
nonumberlist, 						%keine Seitenzahlen anzeigen
nopostdot,							%keine Punkte
style=super,						% Style
acronym,     						%ein Abkkürzungsverzeichnis erstellen
toc,          						%Einträge im Inhaltsverzeichnis
section=chapter]      				%im Inhaltsverzeichnis auf section-Ebene erscheinen
{glossaries}

%custom packages + commands
% Packages
\usepackage{todonotes}
\usepackage{tabularx}
\usepackage{listliketab}
\usepackage{rotating}
\usepackage{listings}
\usepackage{tikz}
\usepackage{titlesec}