%% ============================================================
%%  ***  Page layout
%% ============================================================
\geometry{% 									
left = 2.5cm, 
right=2.5cm, 
top=1.4cm, 
bottom=1.1cm,
includeheadfoot,								
headsep = \dimexpr2\baselineskip-3mm\relax,		% Abstand der Kopfzeile zum Kontext
footskip = \dimexpr2\baselineskip+4mm\relax,	% Abstand der Fußzeile zum Kontext
bindingoffset=1.5mm, 							% max. halb so groß wie der Buchrücken
%showframe										% Rahmen einblenden
}
\KOMAoptions{parskip=yes}						% keine Einrückungen

%% ============================================================

% korrekter Zeilenabstand - \MSonehalfspacing oder \MSdoublespacing wählbar
% anstatt \singlespacing oder \doublespacing
\makeatletter
\newcommand{\MSonehalfspacing}{%
	\setstretch{1.44}%  default
	\ifcase \@ptsize \relax % 10pt
	\setstretch {1.448}%
	\or % 11pt
	\setstretch {1.399}%
	\or % 12pt
	\setstretch {1.433}%
	\fi
}
\newcommand{\MSdoublespacing}{%
	\setstretch {1.92}%  default
	\ifcase \@ptsize \relax % 10pt
	\setstretch {1.936}%
	\or % 11pt
	\setstretch {1.866}%
	\or % 12pt
	\setstretch {1.902}%
	\fi
}
\makeatother
%\MSonehalfspacing	%Gewählte Option

%% ============================================================

% Workaround für römische Zahlen im Inhaltsverzeichen
% Besonder für große Zahlen die viel Platz einnehmen
\makeatletter% --> De-TeX-FAQ
\renewcommand*{\@pnumwidth}{3em}
\makeatother% --> \makeatletter


%% ============================================================
%%  ***  Header & Footer Layout
%% ============================================================

\KOMAoptions{headsepline=true,	% header line
	footsepline=false,			% footer line
	cleardoublepage=plain,	% set empty pages to style 'plain'
	plainheadsepline=false,	% activate header line for plain pages
	plainfootsepline=false}	% activate footer line for plain pages


\pagestyle{scrheadings}
\clearscrheadfoot

% The following checks whether \headmark has the width of 0pt, and if so, changes the color of the headsepline to white:
\newcommand*{\specialheadmark}{%
	\setbox0\hbox{\headmark}%
	\ifdim\wd0=0pt\relax%
	\global\setkomafont{headsepline}{\color{white}}%
	\else%
	\global\setkomafont{headsepline}{\color{black}}%
	\fi%
	\unhbox0%
}

% Setzt Textstill im Footer auf normal - kein Kursiv mehr!
\renewcommand*{\footfont}{\normalfont}

\lehead{\specialheadmark}
\rohead{\specialheadmark}
\ofoot*{\pagemark}


% Redeclare Chapter/Sections/... spacing
\RedeclareSectionCommand[beforeskip=1sp, afterskip=10pt]{chapter}
\RedeclareSectionCommands[beforeskip=1sp, afterskip=1sp]{section,subsection,subsubsection}

\renewcommand{\dateseparator}{.}	% Replace Seperator / by .

%% ============================================================
%%  ***  Reference style
%% ============================================================
\captionsetup{tablewithin=chapter}	% Change 'Table 12' to 'Table 2.3' format - as requested by ILS
\captionsetup{figurewithin=chapter}	% Change 'Figure 12' to 'Figure 2.3' format - as requested by ILS

\setcounter{secnumdepth}{3} % Adjust section numbering here
\setcounter{tocdepth}{4}	% Adjust table of contents depth here



%% ============================================================
%%  ***  Glossary Erstellung
%% ============================================================
%Ein eigenes Symbolverzeichnis erstellen
\newglossary[slg]{symbolslist}{syi}{syg}{Symbolverzeichnis}
% Zusätzliches Feld - Einheit - für das Symbolverzeichnis 
\glsaddkey{unit}{\glsentrytext{\glslabel}}{\glsentryunit}{\GLsentryunit}{\glsunit}{\Glsunit}{\GLSunit}
% Option um SI befehle zu nutzen
\glssetnoexpandfield{unit}
%Den Punkt am Ende jeder Beschreibung deaktivieren
\renewcommand*{\glspostdescription}{}
%Glossar-Befehle anschalten
\makeglossaries

% Neuer Style für das Symbolverzeichnis auf Basis des "long3col" Style
%% ============================================================
\newglossarystyle{symbunitlong}{%
	\setglossarystyle{long3col}% base this style on the list style
	\renewenvironment{theglossary}{% Change the table type --> 3 columns
		\begin{longtable}{@{}l l p{0.8\glsdescwidth} @{}c}}%
		{\end{longtable}}%
	%
	\renewcommand*{\glossaryheader}{%  Change the table header
		\bfseries Symbol & \bfseries Description & & \bfseries Unit \\
		\hline
		\endhead}
	\renewcommand*{\glossentry}[2]{%  Change the displayed items
		\glstarget{##1}{\glossentryname{##1}} %
		& \glossentrydesc{##1}% Description
		&
		& \glsunit{##1}  \tabularnewline
	}
}
%% ============================================================
%Ende des neuen Styles


%% ============================================================
%%  ***  Custom Item Erstellung für Requirements and Deliverables
%% ===========================================================
% eigener Zähler für Requirements
% Ausgabe mit \requirement und \subrequirement
\newcounter{req}
\newcounter{subreq}[req]

\renewcommand\thesubreq{\thereq.\arabic{subreq}}

\newcommand{\requirement}[1]{%
	REQ~\refstepcounter{req}\thereq~#1}

\newcommand{\subrequirement}[1]{%
	REQ~\refstepcounter{subreq}\thesubreq~#1}


% eigener Zähler für Deliverables
% Ausgabe mit \deliverable und \subdeliverable
\newcounter{del}
\newcounter{subdel}[del]

\renewcommand\thesubdel{\thedel.\arabic{subdel}}

\newcommand{\deliverable}[1]{%
	DEL~\refstepcounter{del}\thedel~#1}

\newcommand{\subdeliverable}[1]{%
	DEL~\refstepcounter{subdel}\thesubdel~#1}
