\addchap*{Abstract}

{\LARGE \thetitle}

The development of aviation systems is subject to various requirements, which specify how safe they must be.
Since individual integrated components often do not meet the required probabilities of failure, various concepts,
such as redundancies, are needed to meet certification requirements.
Due to the very theoretical and complex nature of the issue, it is a challenge to raise awareness and potentially develop
long-term interest in the subject, especially among younger target groups.

The goal of this work is the development and implementation of a computer game with an educational background to make the
various concepts visually and interactively accessible.
The game is intended to be used primarily at the \textit{Science Day} and possibly other public events at the
\textit{University of Stuttgart} or the \textit{Institute for Aircraft Systems}.

The basis for the game is the development of a game engine, which is implemented using various common concepts for
game design and object-oriented programming, such as Entity-Component-Systems and class hierarchies.
The application itself implements this engine and is a puzzle game in which the player must build and connect
various system components to configure a functioning system with a sufficiently low probability of failure.
The probability of the occurrence of different error states is calculated using a Markov process.

Through various game levels, gameplay mechanics and aviation system concepts are progressively introduced so that the
player can be gradually introduced to the subject matter.
Furthermore, the generic implementation of the game engine provides a basis with easily maintainable and highly flexible
program code, which can quickly be supplemented with additional features.
