\addchap*{Kurzzusammenfassung der Abschlussarbeit}

{\LARGE Entwicklung einer Gamified Simulation von sicherheitsrelevanten Systemarchitekturen}
%What is your paper about?
Die Entwicklung von Luftfahrtsystemen unterliegt verschiedenen Auflagen, welche angeben, wie sicher sie sein müssen.
Da einzelne integrierte Komponenten geforderte Ausfallwahrscheinlichkeiten oft nicht erfüllen, benötigt es verschiedene
Konzepte wie etwa Rendundanzen um den Anforderungen zur Zertifizierung dennoch entsprechen zu können.
Aufgrund der sehr theoretischen und komplexen Problematik ist es eine Herausforderung, auf das Themengebiet speziell bei
jüngeren Zielgruppen aufmerksam zu machen und gegebenenfalls ein längerfristiges Interesse zu entwickeln.

Ziel dieser Arbeit ist die Entwicklung und Implementierung eines Computerspiels mit Lernhintergrund,
um die verschiedenen Konzepte visuell und interaktiv zugänglich zu machen.
Genutzt werden soll das Spiel vor allem am Tag der Wissenschaft und gegebenenfalls anderen öffentlichen Veranstaltungen der \textit{Universität Stuttgart}
oder dem \textit{Institut für Luftfahrtsysteme}.

Die Grundlage für das Spiel basiert auf der Entwicklung einer Game Engine, welche mithilfe verschiedener üblicher Konzepte
für Game Design und Objekt-orientierte, wie etwa Entity-Component-Systems und Klassenhierarchien, implementiert ist.
Die Applikation selbst implementiert diese Engine und ist ein Puzzlespiel, bei dem der/die Spieler:in verschiedene Systemkomponenten bauen und verbinden
muss, um ein funktionierendes System mit ausreichend niedriger Ausfallwahrscheinlichkeit zu konfigurieren.
Mittels eines Markov Prozesses wird die Wahrscheinlichkeit für das Auftreten verschiedener Fehlerstatus berechnet.

Mittels verschiedener Spiellevel werden progressiv Spielmechaniken und Luftfahrtsystem-Konzepte eingeführt, sodass
der/die Spieler:in langsam an das Themengebiet herangeführt werden kann.
Weiterhin ist durch die generische Implementierung der Game Engine ist eine Basis mit leicht wartbarem und hochflexiblem
Programmcode gegeben, welcher schnell durch weitere Features ergänzt werden kann.
Mittels eines Level-Editors können neue Level erstellt werden, um weitere Konzepte einzuführen.

Eine Gamepad-Integration für das Spiel mit 8-Bit Retro-Grafikstil soll dafür sorgen, dass der/die Spieler:in ein
möglichst interaktives und unterhaltsames Spielerlebnis hat, um so Aufmerksamkeit für das Thema zu schaffen und
einen Lerneffekt durch direktes Feedback zu erzielen.
