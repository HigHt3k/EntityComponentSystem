\section{Basics}\label{sec:basics}
In this chapter, some basic knowledge in order to understand the context and classify this work as a whole will be provided.
The definitions of the most common terms used for aircraft system architecture and the formulas for calculating the relevant
parameters are shown.~\ref{subsec:definitions-system-architecture}
Since this work aims to implement a game, some basic understanding of games and the game development process is described in
section~\ref{subsec:game-design}.
\subsection{Definitions System Architecture}\label{subsec:definitions-system-architecture}
\subsubsection{Failure Probability}
The probability of a failure of a system during a specified point in time is defined as~\cite{lfs2}:
\begin{equation}
    \label{eq:failure-probability}
    P_f(t) \in [0,1]
\end{equation}
\subsubsection{Reliability}
Reliability defines the probability of a working system at a specified point in time,
therefore it is the inverse of the failure probability~\cite{lfs2}:
\begin{equation}
    \label{eq:reliability}
    P_k(t) = 1 - P_f(t) \in [0,1]
\end{equation}
\subsubsection{Integrity}
The detection probability of an error in a system is defined as integrity~\cite{lfs2}:
\begin{equation}
    \label{eq:integrity}
    C \in [0,1]
\end{equation}
\subsubsection{Safety Effect}
There exist different categories of safety effects in the CS-25 for aircraft certification that define the requirements to different systems.
A mapping of safety effect to minimum failure probability has to be achieved in order to successfully certify aircraft systems.
The below table~\ref{tab:safety-effect} displays the different categories.
\begin{table}[!htb]
    \centering
    \begin{tabular}{l|l|l}
        Safety Effect    & Safety Effect short & Accepted Failure Probability \\ \hline
        Catastrophic     & CAT                 & $P_f(1h) <= 10e-9$           \\
        Hazardous        & HAZ                 & $P_f(1h) <= 10e-7$           \\
        Major            & MAJ                 & $P_f(1h) <= 10e-5$           \\
        Minor            & MIN                 & $P_f(1h) <= 10e-3$           \\
        No Safety Effect & NSE                 & $P_f(1h) < 1$
    \end{tabular}
    \caption{Definition of safety effect categories}
    \label{tab:safety-effect}
\end{table}

\subsection{Redundancy Concepts}\label{subsec:redundancy-concepts}
By default, the failure probability of integrated components is defined as $10e-4$, which means, that with only a single component
safety effect requirements of categories higher than \textit{Major} are out of scope.
Therefore, different redundancy concepts have been developed and are being used in system architectures.
These include duplex, triplex and quadruplex systems - meaning, a component is replicated multiple times to run the same program.
One needs to divide between replicated computers, also called \textbf{channels}, and replicated components (e.g.\ CPU) within a computer,
namely \textbf{lanes}.
It is important to note, that each lane or channel is merged right before an actuator.
The decision on how the actuator needs to move is done by a \textbf{voting} component, which can have different methods of voting
for the correct output: mean value, median value, democratic decision.
The voting component generally is the bottleneck of aircraft systems and needs to guarantee a low probability of failure
due to being the most critical component in the chain, since all further movement is based on its decision.
Actuators can also have a monitoring component, which ensures the correct movement of itself.
This is called a \textbf{Com-Mon} - Command \& Monitoring - system.
It negatively affects the failure probability, but the integrity, meaning that the system is able to react to a possible failure due
to noticing it.
A number of different concepts is shown in the figures below, which also serves as a guideline to the level design of the game,
since the presentation and understanding of exactly these concepts is the core of the game.
\subsection{Game Design}\label{subsec:game-design}
Game design is the conceptional process taking place at the very beginning of game development.
Tasks such as defining a basic game idea and laying out the game mechanics, describing the different components of the game and
reiterating those during the development are part of a game designers' field of exercise~\cite{10.5555/2544002}.
\subsubsection{Educational Games}\label{subsubsec:educational-games}
Most games have a certain factor of education, so called \textit{stealth-education} involved, even though not
specifically intended to be educational.
However, there exist also games with the specified purpose of education, this may include simulations, persuasive games,
games for studying and games that support health and growth.
Educational games are a sub-genre of the \textit{Serious Games}-genre and aim to present or solve real world problems,
while maintaining a factor of entertainment.
Different ways of presentation may be used for educational games~\cite[p.43]{10.5555/2544002}.