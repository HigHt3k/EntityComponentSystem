\section{Basics}\label{sec:basics}
In this chapter, some basic knowledge in order to understand the context and classify this work as a whole will be provided.
The definitions of the most common terms used for aircraft system architecture and the formulas for calculating the relevant
parameters are shown.~\ref{subsec:definitions-system-architecture}
Since this work aims to implement a game, some basic understanding of games and the game development process is described in
section~\ref{subsec:game-design}.
\subsection{Definitions System Architecture}\label{subsec:definitions-system-architecture}
\subsubsection{Failure Probability}
The probability of a failure of a system during a specified point in time is defined as~\cite{lfs2}:
\begin{equation}
    \label{eq:failure-probability}
    P_f(t) \in [0,1]
\end{equation}
\subsubsection{Reliability}
Reliability defines the probability of a working system at a specified point in time,
therefore it is the inverse of the failure probability~\cite{lfs2}:
\begin{equation}
    \label{eq:reliability}
    P_k(t) = 1 - P_f(t) \in [0,1]
\end{equation}
\subsubsection{Integrity}
The detection probability of an error in a system is defined as integrity~\cite{lfs2}:
\begin{equation}
    \label{eq:integrity}
    C \in [0,1]
\end{equation}
\subsubsection{Safety Effect}
There exist different categories of safety effects in the CS-25 for aircraft certification that define the requirements to different systems.
A mapping of safety effect to minimum failure probability has to be achieved in order to successfully certify aircraft systems.
The below table~\ref{tab:safety-effect} displays the different categories.
\begin{table}[!htb]
    \centering
    \begin{tabular}{l|l|l}
        Safety Effect    & Safety Effect short & Accepted Failure Probability \\ \hline
        Catastrophic     & CAT                 & $P_f(1h) <= 10e-9$           \\
        Hazardous        & HAZ                 & $P_f(1h) <= 10e-7$           \\
        Major            & MAJ                 & $P_f(1h) <= 10e-5$           \\
        Minor            & MIN                 & $P_f(1h) <= 10e-3$           \\
        No Safety Effect & NSE                 & $P_f(1h) < 1$
    \end{tabular}
    \caption{Definition of safety effect categories}
    \label{tab:safety-effect}
\end{table}

\subsection{Redundancy Concepts}\label{subsec:redundancy-concepts}
By default, the failure probability of integrated components is defined as $10e-4$, which means, that with only a single component
safety effect requirements of categories higher than \textit{Major} are out of scope.
Therefore, different redundancy concepts have been developed and are being used in system architectures.
These include duplex, triplex and quadruplex systems - meaning, a component is replicated multiple times to run the same program.
One needs to divide between replicated computers, also called \textbf{channels}, and replicated components (e.g.\ CPU) within a computer,
namely \textbf{lanes}.
It is important to note, that each lane or channel is merged right before an actuator.
The decision on how the actuator needs to move is done by a \textbf{voting} component, which can have different methods of voting
for the correct output: mean value, median value, democratic decision.
The voting component generally is the bottleneck of aircraft systems and needs to guarantee a low probability of failure
due to being the most critical component in the chain, since all further movement is based on its decision.
Actuators can also have a monitoring component, which ensures the correct movement of itself.
This is called a \textbf{Com-Mon} - Command \& Monitoring - system.
It negatively affects the failure probability, but the integrity, meaning that the system is able to react to a possible failure due
to noticing it.
A number of different concepts is shown in the figures below, which also serves as a guideline to the level design of the game,
since the presentation and understanding of exactly these concepts is the core of the game.

\subsection{Game Design}\label{subsec:game-design}
Game design is the conceptional process taking place at the very beginning of game development.
Tasks such as defining a basic game idea and laying out the game mechanics, describing the different components of the game and
reiterating those during the development are part of a game designers' field of exercise~\cite{10.5555/2544002}.

\subsubsection{Educational Games}\label{subsubsec:educational-games}
Most games have a certain factor of education, so called \textit{stealth-education} involved, even though not
specifically intended to be educational.
However, there exist also games with the specified purpose of education, this may include simulations, persuasive games,
games for studying and games that support health and growth.
Educational games are a sub-genre of the \textit{Serious Games}-genre and aim to present or solve real world problems,
while maintaining a factor of entertainment.
Different ways of presentation may be used for educational games~\cite[p.43]{10.5555/2544002}.

\subsubsection{Purpose of the Game}\label{subsubsec:purpose}
The purpose of this thesis is to provide a game that can be used as a demonstrator at different events of the Institute for Aircraft
Systems at the University of Stuttgart, such as the `Unitag'.

\subsection{Markov Process}\label{subsec:markov-process}
The aircraft design process generally has to comply with the requirements that are stated in regulatory files such as the FAR 25
in order to qualify for the aircraft certification documents.
There are different guidelines for the conduction of the safety assessment process for aircraft systems, which include the
recommendation of quantitative analysis methods such as Fault Tree Analysis, Dependence Diagrams and Markov Analysis.
While the most widely used method today is the Fault Tree Analysis~\cite{7447967}, the Markov Analysis is used for the backend
part of this work and will therefore be explained in detail.

A Markov process, also called Markov chain, is a mathematical model describing a system that changes over time.
Future states of the model are determined only by its current state and by the transition probability, not taking into account
the former states. \todo{add source}

\subsubsection{States}\label{subsubsec:states}
Possible states for the system design Markov model include the correct state and different failure states, such as out of control
or passive failures.
Each failure can also be of a different kind, e.g. a mechanical failure has a different probability than a loss of electrical power.

\subsection{Entity Component System}\label{subsec:entity-component-system}
For the game engine, a generic entity component system (ECS) design pattern is used.
The basic idea behind using an ECS is to separate object data and behavior into components, respectively systems.
Each entity is a unique identifier for an object within the game, this can be anything from a button to a sound.
Entities consist of a variety of components, while each entity can have different components based on the needs.
A purely graphical entity which shows an image may only consist of a single graphics component, however another entity may contain
a component for graphics, one for sound and another one that holds collision information.
It is important to note, that components never change an entities' behavior, but only contain data that describes the entity.
Systems are responsible for collecting all relevant entities (i.e.\ entities that contain specific components) and processing them.
An example for a system is a rendering engine, which collects all graphical entities \& components and renders them to the screen by
using the data that is stored within the respective graphics components.
\\
There are different approaches on the implementation of Entity Component Systems, and they may be combined with other programming
patterns.