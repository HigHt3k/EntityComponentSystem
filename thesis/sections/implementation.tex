\section{Implementation}\label{sec:implementation}
In this chapter, the implementation of the game engine including the basic framework, graphic rendering and input handling is described.
The implementation of the features of the game itself are shown and the connections between different components is described by using a
UML class diagram.
\subsection{Programming Language}\label{subsec:programming-language}
The programming language that was chosen up front is \textit{JAVA}.
This decision was made based on the fact that it makes sense to have a) and object-oriented programming language in order to
handle entities easily and b) because it is possible to run the program on almost every system due to \textit{JAVA} running
in a virtual machine.
\subsection{Game Engine Development}\label{subsec:game-engine-development}
\subsubsection{Entity Component System}
\subsection{Example Markov Chains}\label{subsec:example-markov-chains}
In the following chapter, some examples for Markov chains that reflect the level designs will be shown and the implementation of the
MarkovProcessor will be explained by using these graphs.

\begin{center}
    \begin{tikzpicture}[->, >=stealth', auto, semithick, node distance=3cm]
    \tikzstyle{every node}=[scale=0.7]
    \node (A) {\TBox[fill=white]{C}\TBox[fill=white]{C}\TBox[fill=white]{C}};
    \node (C)[below of=A] {\TBox[fill=white]{C}\TBox[fill=white]{C}\TBox[fill=white]{C}};
    \node (B)[left of=C] {\TBox[fill=white]{C}\TBox[fill=white]{C}\TBox[fill=white]{C}};
    \node (D)[right of=C] {\TBox[fill=white]{C}\TBox[fill=white]{C}\TBox[fill=white]{C}};
    \node (E)[below left of=B] {\TBox[fill=white]{C}\TBox[fill=white]{C}\TBox[fill=white]{C}};
    \node (F)[right of=E] {\TBox[fill=white]{C}\TBox[fill=white]{C}\TBox[fill=white]{C}};
    \node (G)[right of=F] {\TBox[fill=white]{C}\TBox[fill=white]{C}\TBox[fill=white]{C}};
    \node (H)[right of=G] {\TBox[fill=white]{C}\TBox[fill=white]{C}\TBox[fill=white]{C}};
    \node (I)[right of=H] {\TBox[fill=white]{C}\TBox[fill=white]{C}\TBox[fill=white]{C}};
    \node (J)[right of=I] {\TBox[fill=white]{C}\TBox[fill=white]{C}\TBox[fill=white]{C}};
    \path
    (A) edge (B)
        (A) edge (C)
    (A) edge (D)
    \end{tikzpicture}
\end{center}