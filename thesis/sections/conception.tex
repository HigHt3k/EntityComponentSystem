\section{Conception}\label{sec:conception}
\subsection{Background}\label{subsec:background}

\subsection{Genre}\label{subsec:genre}
The \textit{Puzzle} genre is chosen for the graphical user interface and interaction, while the major calculations for the
games' content may be categorized as \textit{Simulation}.

\subsection{Intended Gameplay}\label{subsec:intended-gameplay}
The gameplay is described for a singular level, however the basic concept is the same for each level.
A level usually describes an aircraft system, the goal of the game is to finish or repair a given system.
In order to do that, the user needs to choose from a variety of given components to build the system in a way, that the
required safety level will be reached.
This can be done in different ways, however the efficiency (i.e.\ how many components were used in the building process) determine
the rating (1 star, 2 stars, 3 stars) upon passing a level.
Upon starting a level, the user will face a grid-based puzzle with some set components such as sensors, computers, actuators
or cables.
The user may add new components by choosing them from a limited stack of given components.
Components cannot interfere with other components, however components may also be removed or replaced.
As soon as the user presses a button to simulate the system, a simulation process in the backend is triggered to calculate
the safety rating of the system built.
Based on the safety rating calculated, an animation is triggered that shows the performance of the user visually by starting, cruising
and landing a plane.
If the safety rating is too low, the level is not passed and can be replayed or modified, if the rating is above the given
threshold, the level is passed and another level can be chosen.

\subsection{Game Mechanics}\label{subsec:game-mechanics}

\subsection{Graphic Design}\label{subsec:graphic-design}
For graphics design, the graphic programs \textit{Krita} \cite{foundation_2020} and \textit{Gimp} \cite{gimp} are used.
\subsubsection{List of components}
The following components need to be designed:
\begin{itemize}
    \item Screen backgrounds
    \item Background tilesq
    \item Simulation components: actuator, sensor, computer, com-mon, cable, \ldots
    \item Build panel
    \item Requirement panel
\end{itemize}

\subsection{Sound Design}\label{subsec:sound-design}
Sound design is conducted with the DAW (digital audio workstation) software \textit{FL Studio 12} \cite{imageline}.